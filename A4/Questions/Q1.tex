\section{Problem Statement}
\textit{Outline the problem statement (5), the mathematical model (5), and the computational model parameters (5). In your report, provide the geometry of the simulation domain, governing equations, and mathematical formulation of boundary and initial conditions. In computational model section, present the mesh of the simulation domain (only one mesh out of five), the values of boundary and initial conditions, convergence criteria is set for RMS of momentum at $10^{-5}$, provide the name and order of the scheme used for convective term discretization (should be second order accurate). Use the coupled solver. In a table, list all the models that you selected in “Methods” tab of the software, as well as the details of the solver under “General”.}

\subsection{Objective}
The problem objective is to calculate the drag coefficient $C_d$ of a square object at a Reynolds number of 20 and estimate the length of a vortex that forms behind the object. 

Secondary objectives include learning about mesh refinement and mesh convergence. The problem will be solved numerically in ANSYS Fluent for five systematically refined meshes. The first (coarsest) mesh should contain in the order of 2000-3000 cells or control volumes with a cell size of 5.0 mm. The rest of the meshes are generated by reducing the size of each cell by two in both directions, i.e. increasing the total number of cells by four. The topology of the mesh remains the same as the mesh is refined. During the simulation, it is necessary to monitor the drag force $F_d$ as a function of iteration.

\subsection{Mathematical Model}
put NS in here
put BC in here
put IC in here

\subsection{Computational Model Parameters}
First $U_\infty$ needs to be determined. The Reynolds number is given from the problem statement as $Re = 20$. Then,
\begin{align*}
    \text{Re} &= \frac{U_\infty \rho D}{\mu} \\ 
    \implies U_\infty &= \frac{\text{Re} \mu}{\rho D}
\end{align*}
The properties of air are given as $\rho = 1.0$ kg/m$^3$ and $\mu = 2.0 \times 10^{-5}$ kg/m$\cdot$s. The side length of the square object is $D = 0.01$ m. Thus,
\begin{align*}
    U_\infty &= \frac{20 \times 2.0 \times 10^{-5}}{1.0 \times 0.01} = 0.04 \text{ m/s}
\end{align*}

The mesh element size will be 5.0 mm for the first mesh. The mesh will be refined by reducing the mesh element size by 2, resulting in a total of 4 times the number of cells. The mesh will be refined 5 times. The convergence criteria is set for the RMS of momentum at $10^{-5}$. The scheme used for convective term discretization is second order accurate. The coupled solver will be used.

5 inflation layers, growth rate of 1.2, default transition ratio 0.272

use viscous laminar model 

describe the BC i guess idk what they want

\begin{table}[h]
    \centering
    \caption{General Solver Details}
    \begin{tabular}{cc}
        \toprule
        Type & Pressure Based \\
        Velocity Formation & Absolute \\
        Time & Steady \\
        2D Space & Planar \\
        \bottomrule
    \end{tabular}
\end{table}
    method: coupled
    Gradient: least squares cell based
    Pressure: Second order
    Momentum: Second order upwind

    start point (0.001, 0.005)
    end point (0.20005, 0.005)
    max iterations 1000

    mesh 1: 2390 elements
    mesh 2: element size 0.0025, 9390 elements
    mesh 3: element size 0.00125, 37870 elements
    mesh 4: element size 0.000625, 152044 elements
    mesh 5: element size 0.0003125, 613378 elements