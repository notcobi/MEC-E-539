\section{}
Consider the general form of momentum balance
\begin{align}
    \rho \frac{dv_i}{dt} = \frac{\partial T_{ji}}{\partial x_j} + \rho b_i \label{eq:1}
\end{align}
vhere d/dt is the total derivative; $v_i$ is the velocity; 
$\rho$ denotes density; $T_{ij}$ is the stress tensor and $b_i$ is a body force. 
This equation says that inertial and body forces are balanced by the gradients of the stress tensor.

The general form of the stress tensor reads
\begin{align}
    T_{ij} = -p \delta_{ij} + \mu \left(\frac{\partial v_i}{\partial x_j} + 
    \frac{\partial v_j}{\partial x_i} \right) - \frac{2}{3} \mu 
    \frac{\partial v_k}{\partial x_k} \delta_{ij} + \kappa \left(
    \frac{\partial v_k}{\partial x_k} \delta_{ij} \right) \label{eq:2}
\end{align}
vhere $p$ is the pressure, $\mu$ is the shear (dynamic) viscosity and $\kappa$ is the 
dialation viscocity. For each question assume the viscosities $\mu$ and $\kappa$ are constant.

\begin{enumerate}[label=(\alph*)]
    \item Using equations (1) and (2) derive the general form of Navier-Stokes equations for
    compressible viscous fluid. Note: only index notation can be used. The final form of
    equations should be presented in operator (vector) form.
    \item Assuming $\kappa = 0$, vrite the full set of Navier-Stokes equations
    in Cartesian coordinates assuming tvo-dimensional flov (tvo equations corresponding to $x$ and $y$
    directions).
\end{enumerate}

\subsection*{Solution}
\subsection{}
First, deal vith the stress tensor.
\begin{align*}
    T_{ji} = -p \delta_{ji} + \mu \left(\frac{\partial v_j}{\partial x_i} +
    \frac{\partial v_i}{\partial x_j} \right) - \frac{2}{3} \mu
    \frac{\partial v_k}{\partial x_k} \delta_{ji} + \kappa \left(
    \frac{\partial v_k}{\partial x_k} \delta_{ji} \right)
\end{align*}
Differentiate vith respect to $x_j$,
\begin{align*}
    \frac{\partial T_{ji}}{\partial x_j} &= -\frac{\partial p}{\partial x_j} \delta_{ji}
    + \mu \left(\frac{\partial^2 v_j}{\partial x_j \partial x_i} + 
    \frac{\partial^2 v_i}{\partial x_j^2} - 
    \frac{2}{3} \frac{\partial^2 v_k}{\partial x_j \partial x_k} \delta_{ji}\right) + 
    \kappa \left(\frac{\partial^2 v_k}{\partial x_j \partial x_k} \delta_{ji} \right)  
\end{align*}
Substituting into equation \ref{eq:1},
\begin{align*}
    \rho \frac{\partial v_i}{\partial t} + \rho v_j \frac{\partial v_i}{\partial x_j} 
    &= -\frac{\partial p}{\partial x_j} \delta_{ji}
    + \mu \left(\frac{\partial^2 v_j}{\partial x_j \partial x_i} + 
    \frac{\partial^2 v_i}{\partial x_j^2} - 
    \frac{2}{3} \frac{\partial^2 v_k}{\partial x_j \partial x_k} \delta_{ji}\right) + 
    \kappa \left(\frac{\partial^2 v_k}{\partial x_j \partial x_k} \delta_{ji} \right) + \rho b_i \\
    &= -\frac{\partial p}{\partial x_i} + \mu \left(\frac{\partial^2 v_j}{\partial x_j \partial x_i} +
    \frac{\partial^2 v_i}{\partial x_j^2} - \frac{2}{3} \frac{\partial^2 v_k}{\partial x_i \partial x_k} 
    \right) + \kappa \left(\frac{\partial^2 v_k}{\partial x_i \partial x_k} \right) + \rho b_i
\end{align*}
Converting the RHS to vector form,
\begin{align*}
    \text{RHS} &=
    -\nabla p + \mu \left[\nabla (\nabla \cdot \vec{v}) + \nabla^2 \vec{v} - 
    \frac{2}{3} \nabla (\nabla \cdot \vec{v}) \right] 
    + \kappa \nabla (\nabla \cdot \vec{v}) + \rho \vec{b} \\
    &= -\nabla p + \mu \nabla^2 \vec{v} 
    + \left[\frac{\mu}{3} + \kappa\right] \nabla (\nabla \cdot \vec{v}) + \rho \vec{b}
\end{align*}
Converting the LHS to vector form,
\begin{align*}
    \text{LHS} = \rho \left(\frac{\partial \vec{v}}{\partial t} 
    + (\vec{v} \cdot \nabla) \vec{v} \right)
\end{align*}
Combining the LHS and RHS,
\begin{align*}
    \boxed{
    \rho \left(\frac{\partial \vec{v}}{\partial t}
    + (\vec{v} \cdot \nabla) \vec{v} \right) = -\nabla p + \mu \nabla^2 \vec{v}
    + \left[\frac{\mu}{3} + \kappa\right] \nabla (\nabla \cdot \vec{v}) + \rho \vec{b}
    }
\end{align*}

\subsection{}
Assuming $\kappa = 0$, the equation reduces to 
\begin{align*}
    \rho \left(\frac{\partial \vec{v}}{\partial t}
    + (\vec{v} \cdot \nabla) \vec{v} \right) = -\nabla p + \mu \nabla^2 \vec{v}
    + \frac{\mu}{3} \nabla (\nabla \cdot \vec{v}) + \rho \vec{b}
\end{align*}
In Cartesian coordinates (x, y), assuming $\vec{v} = (u, v)$ and $\vec{b} = (b_x, b_y)$.
In the x-direction,
\begin{align*}
    \boxed{
    \rho \left(\frac{\partial u}{\partial t} + u \frac{\partial u}{\partial x} 
    + v \frac{\partial u}{\partial y} \right) = -\frac{\partial p}{\partial x} 
    + \mu \left(\frac{\partial^2 u}{\partial x^2} + \frac{\partial^2 u}{\partial y^2} \right)
    + \frac{\mu}{3} \left(\frac{\partial^2 u}{\partial x^2} + \frac{\partial^2 v}{\partial x \partial y} \right)
    + \rho b_x 
    }
\end{align*}
In the y-direction,
\begin{align*}
    \boxed{
    \rho \left(\frac{\partial v}{\partial t} + u \frac{\partial v}{\partial x} 
    + v \frac{\partial v}{\partial y} \right) = -\frac{\partial p}{\partial y} 
    + \mu \left(\frac{\partial^2 v}{\partial x^2} + \frac{\partial^2 v}{\partial y^2} \right)
    + \frac{\mu}{3} \left(\frac{\partial^2 u}{\partial x \partial y} + \frac{\partial^2 v}{\partial y^2} \right)
    + \rho b_y 
    }
\end{align*}

% Note, expansion of $\nabla (\nabla \cdot \vec{v})$ in Cartesian coordinates,
% \begin{align*}
%     \nabla (\nabla \cdot \vec{v}) &= \nabla \left(\frac{\partial u}{\partial x} + \frac{\partial v}{\partial y} \right) \\
%     &= \begin{bmatrix}
%         \frac{\partial}{\partial x} \left(\frac{\partial u}{\partial x} + \frac{\partial v}{\partial y} \right) \\
%         \frac{\partial}{\partial y} \left(\frac{\partial u}{\partial x} + \frac{\partial v}{\partial y} \right)
%     \end{bmatrix} 
% \end{align*}
% also expand $\nabla \cdot (\nabla \vec{v})$ in Cartesian coordinates,
% \begin{align*}
%     \nabla \cdot (\nabla \vec{v}) &= \nabla \cdot 
%     \begin{bmatrix}
%         \frac{\partial u}{\partial x} & \frac{\partial v}{\partial x} \\
%         \frac{\partial u}{\partial y} & \frac{\partial v}{\partial y}
%     \end{bmatrix} \\
%     &= \begin{bmatrix}
%         \frac{\partial}{\partial x} \left(\frac{\partial u}{\partial x} \right) + \frac{\partial}{\partial y} \left(\frac{\partial v}{\partial x} \right) \\
%         \frac{\partial}{\partial x} \left(\frac{\partial u}{\partial y} \right) + \frac{\partial}{\partial y} \left(\frac{\partial v}{\partial y} \right)
%     \end{bmatrix} \\
% \end{align*}
% we have $\partial_{i} \partial_{j} a_i$