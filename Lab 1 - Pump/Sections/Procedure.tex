\section{Procedure}
% Requirements
% State the procedure followed in taking the data. The procedure should be organized 
% in the most concise, logical order; it does NOT need to be chronological. Use past 
% tense to tell the reader how you made the measurements. The reason for making each 
% measurement should also be given if necessary. e.g. "Stagnation pressure was measured 
% at 12 locations across the duct in order to determine average velocity". Include the make 
% and model number for important equipment used in the study. For MecE301 reports 
% this section is usually one or two paragraphs and one or two schematic figures. 
% Presenting, and referring to, a schematic diagram of the set-up saves words (and time) 
% and helps reader comprehension

\subsection{Equipment}
\noindent The following equipment was used to perform the experiment: Arduino Uno, MEC E 301 Shield, computer, oscilloscope, and jumper wires.

\subsection{Calibration}
\noindent For extensive details on the calibration procedure, refer to the precis provided by the MEC E 301 course
All measurements were taken from the serial monitor of the Arduino IDE. During calibration, different circuit components 
and reference voltages were used to measure the voltage output of the PCB.

First, measuring voltages with a 5V reference voltage was performed. As seen in Appendix \ref{sec:figures}, Figure \ref{fig:noaref}, the \texttt{5V} and \texttt{GND} 
pins on the Arduino Uno were connected to the \texttt{5V\_VIN} and \texttt{GND} pins on the PCB. The \texttt{A0} pin on the Arduino Uno was connected to the output pin,
\texttt{2.500V}. After uploading the sketch, ten values were recorded from the serial monitor. The \texttt{2.500V} output pin was then
swapped to \texttt{1.800V}, \texttt{1.024V}, and \texttt{0.102V} and ten values were recorded for each.

Next, measuring voltages with a 3.3V reference voltage and various circuit components was performed. As shown in Appendix \ref{sec:figures}, Figure \ref{fig:aref},
the AREF jumper on the MEC E 301 Shield was inserted, connecting the 3.3V reference voltage to the AREF pin on the Arduino Uno. First measurements of the voltages \texttt{2.500}, 
\texttt{1.800}, \texttt{1.024}, and \texttt{0.102} were performed with the 3.3V reference voltage. Then the output pins of the PCB were connected to the MEC E 301 Shield input-output 
pins to measure the various circuit components. The input-output pairs of [\texttt{D10\_I}, \texttt{D10\_O}], [\texttt{+/-10\_I}, \texttt{+/-10\_O}], and [\texttt{X10\_I}, 
\texttt{X10\_O}] measured voltages with a voltage divider, [-10, 10]V range, and amplifier respectively. Again, the voltages of \texttt{2.500}, \texttt{1.800}, \texttt{1.024}, 
and \texttt{0.102} were measured and recorded. Note, the amplifier only measured \texttt{0.102} because the other voltages were out of range.


% Next, measuring voltages with a 3.3V reference voltage was performed. The AREF jumper on the MEC E 301 Shield was inserted, connecting the
% 3.3V reference voltage to the AREF pin on the Arduino Uno. The code was modified to reflect the new reference voltage, \texttt{float voltage = sensorValue * (3.3 / 1024.0);}.
% The voltages \texttt{2.500}, \texttt{1.800}, \texttt{1.024}, and \texttt{0.102} were measured again and ten values were recorded for each.

% The next calibrations utilized different circuit components. Connecting the output pin on the PCB to the \texttt{D10\_I}, \texttt{+/-10\_I}, and \texttt{X10\_I} pins on 
% the MEC E 301 Shield allowed for the use of a voltage divider, [-10, 10]V range, and amplifier respectively. The \texttt{D10\_O}, \texttt{+/-10\_O}, and \texttt{X10\_O} 
% pins on the MEC E 301 Shield were connected to the \texttt{A0} pin on the Arduino Uno. The calibration voltages \texttt{2.500}, \texttt{1.800}, \texttt{1.024}, and \texttt{0.102}
% were measured again and ten values were recorded for each.

\subsection{Time Varying Voltage}
\noindent The \texttt{SINE\_OP} pin on the PCB was connected to the \texttt{A0} pin on the Arduino Uno as shown in Appendix \ref{sec:figures}, Figure \ref{fig:aref}
The code was modified to output a timestamp and voltage value and the baud rate was set to 115200. The serial monitor was opened and the voltage was recorded for 255 values. The values were 
copied into an Excel spreadsheet for analysis.

Lastly, measurement using an oscilloscope was performed. A schematic of the system is shown in Appendix \ref{sec:figures}, Figure \ref{fig:oscilloscope}. 
The \texttt{SINE\_OP} pin on the PCB was connected to the oscilloscope.  The \texttt{Autoset} button was pressed to automatically set the oscilloscope. Occasionally,
the \texttt{Autoset} needed to be set to a sawtooth wave to get a good reading. Lastly, the \texttt{Measure} button was pressed and the oscilloscope was set to 
measure the peak-to-peak voltage, frequency, and mean voltage.
% \forceindent The following procedure was followed as outlined in the lab precis \cite{lab2precis}. First, the Arduino Uno was 
% connected to the computer and the Arduino IDE was opened. The Arduino IDE was used to upload the \texttt{AnalogReadSerial} after 
% modifying the code \texttt{float voltage = sensorValue * (5.0 / 1024.0);} to correct the decimal-to-number conversion. 
% An additonal line was added \texttt{Serial.println(voltage, 3)} to print the voltage to the serial monitor with 3 decimal places. 

% First, measuring voltages with a 5V reference voltage was performed. The \texttt{5V} and \texttt{GND} pins on the Arduino Uno were 
% connected to the \texttt{5V\_VIN} and \texttt{GND} pins on the PCB. The \texttt{A0} pin on the Arduino Uno was connected to the output pin,
% \texttt{2.500V}. After uploading the sketch, ten values were recorded from the serial monitor. The \texttt{2.500V} output pin was then
% swapped to \texttt{1.800V}, \texttt{1.024V}, and \texttt{0.102V} and ten values were recorded for each. 

% Next, measuring voltages with a 3.3V reference voltage was performed. The AREF jumper on the MEC E 301 Shield was inserted, connecting the 
% 3.3V reference voltage to the AREF pin on the Arduino Uno. The code was modified to reflect the new reference voltage, \texttt{float voltage = sensorValue * (3.3 / 1024.0);}.
% The voltages \texttt{2.500}, \texttt{1.800}, \texttt{1.024}, and \texttt{0.102} were measured again and ten values were recorded for each.

% % next was using voltage divider to measure voltages
% Next, measuring voltages with a voltage divider was performed. The output pin on the PCB was connected to the \texttt{D10\_I} on the MEC E 301 Shield.
% The \texttt{D10\_O} pin on the MEC E 301 Shield was connected to the \texttt{A0} pin on the Arduino Uno. The code was modified to reflect the new range 
% of voltages, \texttt{float voltage = sensorValue * (33.0 / 1024.0);}. The voltages \texttt{2.500}, \texttt{1.800}, \texttt{1.024}, and \texttt{0.102} 
% were measured again and ten values were recorded for each.

% % using -10 to 10 v range
% Next, measuring voltages with a -10V to 10V range was performed. The \texttt{+/-10\_I} was connected to the PCB output pin. The \texttt{+/-10\_O} pin on the MEC E 301 Shield
% was connected to the \texttt{A0} pin on the Arduino Uno. The code was modified to reflect the new range of voltages, \texttt{float voltage = sensorValue * (20.0 / 1024.0) - 10.0;}.
% The voltages \texttt{2.500}, \texttt{1.800}, \texttt{1.024}, and \texttt{0.102} were measured again and ten values were recorded for each.

% %using amplifier
% Voltage measurements with an amplifier was performed. The \texttt{X10\_I} pin on the MEC E 301 Shield was connected to the PCB output pin. The \texttt{X10\_O} pin on the MEC E 301 Shield
% was connected to the \texttt{A0} pin on the Arduino Uno. The code was modified to reflect the new range of voltages, \texttt{float voltage = sensorValue * (0.33 / 1024.0);}.
% The voltages \texttt{2.500}, \texttt{1.800}, \texttt{1.024}, and \texttt{0.102} were measured again and ten values were recorded for each.

% % pcb time varying voltage
% Time varying voltage measurements were performed. The \texttt{SINE\_OP} pin on the PCB was connected to the \texttt{A0} pin on the Arduino Uno. The code was modified to reflect the new range of voltages, 
% \texttt{float voltage = sensorValue * (3.3 / 1024.0);}. The code was modified to ouptut a timestamp and voltage value and the baud rate was set to 115200. The serial monitor was opened and the
% voltage was recorded for 255 values.


