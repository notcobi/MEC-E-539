\section*{Conservation of Energy (Scalar)}

Energy is a scalar quantity, and therefore it can be identified \& formulated through the conservation of a scalar quantity:

Let's try $\phi = $ scalar,
\begin{align*}
    \frac{\partial}{\partial t} \int_{\text{C.V.}} \rho \phi \, dV 
    + \int_{\text{C.S.}} \rho \phi (\vec{u} \cdot \vec{n}) \, dS &= 
    \oint_{\text{C.S.}} \Gamma \Grad(\phi) \cdot \vec{n} \, dS + 
    \int_{\text{C.V.}} \underbrace{q_{\phi}}_{\text{source/sink}} \, dV
\end{align*}

\begin{align*}
    \implies 
    \frac{\partial}{\partial t} (\rho \phi)  
    + \frac{\partial}{\partial x_j} (\rho \phi u_j) &=
    \frac{\partial}{\partial x_j} (\Gamma \frac{\partial \phi}{\partial x_j}) + q_{\phi}
\end{align*}
Conservation of a scalar (Energy). In vector notation,
\begin{align*}
    \partial_{t} (\rho \phi) + \Div (\rho \phi \vec{u}) &= 
    \Div (\Gamma \Grad (\phi)) + q_{\phi} 
\end{align*}
Where
\begin{itemize}
    \item $\partial_{t} (\rho \phi)$ is the time rate of change of the scalar quantity $\phi$ (conservative term).
    \item $\Div (\rho \phi \vec{u})$ is the rate of change due to the flow due to $\vec{u}$ (advection term).
    \item $\Div (\Gamma \Grad (\phi))$ is the rate of change due to diffusion ($\Gamma$) (diffusion term).
    \item $q_{\phi}$ is the rate of production (source) or destruction (sink) of $\phi$.
\end{itemize}

\section*{Chapter 4. Fundamental flows (Simplification)}
Navier-Stokes equations are highly non-linear PDEs with no exact solutions. However,
there are fundamental flow dynamics (simplified flows) based on assumptions and approximations that 
makes the mathematics easier to follow, solve, and interpret.

We are going to look at 4 simplified flow cases:
\begin{enumerate}
    \item Incompressible flow ($\rho = \text{constant}$)
    \item Invicid flow (Euler's flow) ($\mu \to 0$)
    \item Creeping flow (Stokes flow) ($\text{Re} \ll 100$, inertial forces are negligible)
    \item Potential flow ($\text{Re} \to 0$, $\text{Ma} \to 0$
\end{enumerate}
Let's look at the conservation laws for each:
\subsection*{Incompressible flow}
Incompressibility is defined as incapability of a fluid (i.e. liquid) to compress to a smaller size
under internal/external loads. This, therefore, means that their \textbf{density $\rho$} does
not change as long as we keep their mass the same.

Typically, liquids are incompressible, but air (gas) can become compressible at special conditions.
\[
 \underbrace{\text{Ma}}_{\vec{u} /\text{speed of sound}} > 0.3 \implies \text{compressible}
\]

Continuity:
\begin{align*}
    \cancelto{0 (\text{Incomp})}{\frac{\partial \rho}{\partial t}} + \divergence (\rho \vec{u}) &= 0 \\
    \implies \divergence \vec{u} &= 0
\end{align*}
Momentum:
\begin{align*}
    \frac{\partial}{\partial t} (\rho \vec{u}) + \divergence (\rho \vec{u} \vec{u}) &=
    - \grad P + \cancelto{0, \;\text{continuity}}{\frac{1}{3} \mu \nabla(\divergence \vec{u})} + \mu \laplacian \vec{u} + \rho \vec{b} \\
\end{align*}
Expanding the $\divergence (\rho \vec{u} \vec{u})$ term,
\begin{align*}
    \divergence (\rho \vec{u} \vec{u}) &= (\divergence \rho \vec{u}) \vec{u}
        + \rho \vec{u} \cdot \grad \vec{u} \\
    &= \cancelto{0, \;\text{continuity}}{(\divergence \rho \vec{u})} \vec{u}
        + \rho \vec{u} \cdot \grad \vec{u} \\
    &= \rho \vec{u} \cdot \grad \vec{u}
\end{align*}
Therefore,
\begin{align*}
    \frac{\partial}{\partial t} (\rho \vec{u}) + \rho \vec{u} \cdot \grad \vec{u} &=
    - \grad P + \mu \laplacian \vec{u} + \rho \vec{b} \\
    \rho \frac{\partial \vec{u}}{\partial t} + \rho \vec{u} \cdot \grad \vec{u} &=
    - \grad P + \mu \laplacian \vec{u} + \rho \vec{b} \\
    \frac{\partial \vec{u}}{\partial t} + \vec{u} \cdot \grad \vec{u} &=
    -\frac{\grad P}{\rho} + \underbrace{\frac{\mu}{\rho}}_{\nu} \laplacian \vec{u} + \vec{b} \\
\end{align*}

\subsection*{Invicid flow (Euler's flow)}
Viscous forces can be important in flows close to a wall, where we have large velocity gradients (Also in wakes).
As we should before, it is the combination of $\nu$ and $\vec{u}$ that forms the viscous effects in transport of fluids. 

$\implies$ Vorticies $\to$ $\nu$ may be important.

$\hookrightarrow$ if you are far from a surface or regions of large velocity gradients, the implication of viscocity becomes minimal.

$\hookrightarrow$ we quantify the effect of viscocity in the flow using Reynold's Number:
\begin{align*}
    \text{Re} = \frac{\rho u \overbrace{L}^{\text{characteristic length}}}{\mu} 
    = \frac{\text{inertial forces}}{\text{viscous forces}}
\end{align*}
if Re $\gg 1000 \implies \mu \to 0$ which means inertial forces dominate the flow 
(negligible viscous forces).

Continuity:
\begin{align*}
    \frac{\partial \rho}{\partial t} + \divergence (\rho \vec{u}) &= 0 
\end{align*}
No impact because no $\mu$ term.

Momentum:
\begin{align*}
    \frac{\partial}{\partial t} (\rho \vec{u}) + \divergence (\rho \vec{u} \vec{u}) &=
    - \grad P + \cancelto{0, \;\text{invicid}}{\frac{1}{3}\mu \nabla(\divergence \vec{u})}
    + \cancelto{}{\mu \laplacian \vec{u}} + \rho \vec{b} \\
\end{align*}
\begin{align*}
    \boxed{\frac{\partial}{\partial t} (\rho \vec{u}) + \divergence (\rho \vec{u} \vec{u}) =
    - \grad P + \rho \vec{b}}
\end{align*}
Note: As we saw in the Navier-Stokes equation, the flow can only be dominated by the 
\textbf{Pressure} and \textbf{External forces}. This means that the invicid cond. cannot hold if we 
are dealing with areas of high straining (vorcity and wakes).

Note: Since we are assuring invicid condition, then the flow cannot slow down close to the 
stationary wall. $\implies$ Slip Boundary Condition.

\subsection*{Creeping flow (Stokes flow)}
At high re, we just discussed that viscous effects are negligible. Contrarily, at low 
Re, (Re $\ll 100$) the effect of viscosity \textbf{dominates} the flow. 

$\implies$ Inertial forces are negligible.

\begin{align*}
    \text{Re} \propto \frac{u D}{\nu} 
    \begin{cases}
        \text{Either } u \to 0 \text{and/or} \\
        D \to 0 
    \end{cases}
\end{align*}

Continuity:
\begin{align*}
    \frac{\partial \rho}{\partial t} + \divergence (\rho \vec{u}) &= 0
\end{align*}
No impact.

Momentum:
\begin{align*}
    0 = - \frac{\grad P}{\rho} +  \divergence (\nu \grad \vec{u}) + \vec{b} + \vec{b}
\end{align*}
This type of flow is mostly for porous media coating, or nano-fluidics.

\subsection*{Potential flow}
One of the simpliest flows in fluid mechanics.

Based on two conditions:
\begin{enumerate}
    \item Invicid flow ($\mu \to 0$)
    \item Irrotational flow ($\vec{\omega} \to 0$)
\end{enumerate}
provides approximation for initial flow conditions.